\documentclass{article}
\usepackage[utf8]{inputenc}
\usepackage{latexsym}
\usepackage{amsfonts}
\usepackage{amssymb}
\usepackage{amsmath}
\usepackage[ruled,vlined]{algorithm2e}
\DeclareUnicodeCharacter{2212}{-}


\newcommand{\newpar}{\vspace{.2in} \noindent}
\usepackage[margin=1.2in]{geometry}

\title{COMP3821 Homework 4}
\author{Jason Qin, z5258237}
\date{April 2020}

\usepackage{natbib}
\usepackage{graphicx}

\begin{document}

\maketitle

\section*{Question 1.}
\paragraph{1.1} \textbf{Nature of LP.} \\

\noindent
Since each jewellery item is discrete, the burgular cannot e.g., take 'half' a diamond ring.
Thus burgular must formulate an integer LP problem. Note this is a case of an 0-1 integer LP problem.

\paragraph{1.2} \textbf{Variables.} \\

\noindent
$n$ items.\\
$c_i$ is the value of the $i$th item, $1 \leq i \leq n$ \\
$w_i$ is the weight of the $i$th item. \\
$M$ is the weight capacity of the backpack. \\
$x_i$ is the corresponding $0$-$1$ boolean value signifying whether an item is taken. \\

\paragraph{1.3} \textbf{Constraints.} \\

\noindent
The only constraint is the weight capacity of the backpack; the total weight of
looted items cannot not exceed the capacity,
$$\sum_{i=1}^{n} w_i \leq M$$

\paragraph*{1.4} \textbf{Objective.} \\

\noindent
The aim is to get the most value out of the items looted such that,
$$\sum_{i=1}^{n} c_ix_i = c_1x_1 + c_2x_2 + c_3x_3 + ... + c_nx_n$$

is maximized.

\end{document}
