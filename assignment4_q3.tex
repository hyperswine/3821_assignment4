\documentclass{article}
\usepackage[utf8]{inputenc}
\usepackage{latexsym}
\usepackage{amsfonts}
\usepackage{amssymb}
\usepackage{amsmath}
\usepackage[ruled,vlined]{algorithm2e}
\DeclareUnicodeCharacter{2212}{-}

\usepackage{wrapfig}

\usepackage{mathtools}
\usepackage{booktabs,multicol,multirow}
\usepackage{wasysym}

\newcommand{\newpar}{\vspace{.2in} \noindent}
\usepackage[margin=1.2in]{geometry}

\title{COMP3821 Homework 4}
\author{Jason Qin, z5258237}
\date{April 2020}

\usepackage{natbib}
\usepackage{graphicx}

\begin{document}

\maketitle

\section*{Question 3.}
% research discrete and continous variables a bit more.
\paragraph*{3.1} \textbf{Discrete Variables.} \\

\noindent
$j$, a discrete variable that takes a value from $1,2...10$, and $inv_j$ represents the $j$th investment. \\
$I$, a set containing the opportunities that are undertaken. \\
$x_j$, a value either of $0$ or $1$ denoting whether an investment is undertaken.

\paragraph*{3.2} \textbf{Continuous Variables.} \\

\noindent
$P_j$ is the gross profit obtained after undertaking $inv_j$. \\
$C_j$ is the initial cost required for $inv_j$. \\
$Q$ is the total capital available for investment.

\paragraph*{3.3} \textbf{Constraints.} \\

\noindent
$Q \geq 0$. If $Q = 0$, then the best investment is simply $I = \{\}$. \\
Either $inv_3 \in I$ OR $inv_4 \in I$, but not both. \\
Either $inv_5 \in I$ OR $inv_6 \in I$, but not both. \\
Unless $inv_3 \in I$ OR $inv_4 \in I$, $inv_5 \notin I$ and $inv_6 \notin I$. \\
$I$ must contain elements $inv_g \in V = \{1,2,7,8,9,10\}$ such that, $2 \leq \sum_{inv \in I} inv_g \leq 4$.

\paragraph*{3.4} \textbf{Objective.} \\

\noindent
We want an $I$ such that net-profit(I) is maximized. The net profit for each investment is $N_j = P_j - C_j$, so the goal is to choose some
$I$ containing the maximum $\sum_{N_j \in I} N_j$.
i.e. the goal is to maximize,
%% perhaps try to think holistically about the problem, what can you make it so you're incorporating everything.

$$N_{max} = N_1x_1 + N_2x_2 + N_3x_3 + ... + N_{10}x_{10}$$

for $0 \leq x_j \leq 1$

\paragraph*{3.5} \textbf{Polynomial-Time solvability.} \\
%think about what the problem really is. Is it an integer LP?
%IF it is, then only polynomial time if P=NP.
%ELSE if you think it is a non-integer LP, then it is Yes, regardless of P vs. NP.

%NOTES: No, regardless of P vs. NP - is probably wrong
%       Yes, if P =/= NP. No, otherwise - sounds very wrong.

\noindent
The following choices are selected,

\begin{itemize}
\item Yes, regardless of P vs. NP.
\item Yes, if P $=$ NP. No otherwise.
\end{itemize}

\end{document}
