\documentclass{article}
\usepackage[utf8]{inputenc}
\usepackage{latexsym}
\usepackage{amsfonts}
\usepackage{amssymb}
\usepackage{amsmath}
\usepackage[ruled,vlined]{algorithm2e}
\DeclareUnicodeCharacter{2212}{-}


\newcommand{\newpar}{\vspace{.2in} \noindent}
\usepackage[margin=1.2in]{geometry}

\title{COMP3821 Homework 4}
\author{Jason Qin, z5258237}
\date{April 2020}

\usepackage{natbib}
\usepackage{graphicx}

\begin{document}

\maketitle

\section*{Question 2.}
\paragraph{2.1} \textbf{Variables.} \\

\noindent
$n$ is the number of distinct toys $t_i$ that can be produced. \\
$m$ is the number of factories $f_j$ availiable for producing toys. \\
$C_i$ is the setup cost for a new factory $i$. \\
$P_i$ is the profit for selling each toy $i$. \\
$H_j$ is the total production hours for factory $j$, $h_j$ is the hours used. \\
$A_{ij}$ is the production rate (toys/hour) of toy $i$ at factory $j$. \\
$S$ is the map of factories to tuples containing the number of toys e.g. (toy $1$: 4, toy $2$: 5).

\paragraph*{2.2} \textbf{Constraints.} \\

\noindent
$(\sum_{i=1}^{n}P_i) - C_j > 0$ to make a net profit at factory $j$. \\
$h_j \leq H_j$ for all $0 \leq j \leq m$.

\paragraph*{2.3} \textbf{Objective.} \\

\noindent
Maximize the net profits per factory to get the maximum total profit from selling toys.
Hence the goal is to maximize,

$$\sum_{s_j \in S}^{m} profit(s_j)$$

where profit($s$) sums up
the total net profit of toys made at a factory.

\paragraph*{2.4} \textbf{Polynomial-Time Solution.} \\

\noindent
Since this is an integer LP problem, it is fundamentally NP-hard with no deterministic polynomial-time solution.
However, a polynomial solution may exist for specific instances of the integer LP problem. This could be one, although, it is not known.


\end{document}
