\documentclass{article}
\usepackage[utf8]{inputenc}
\usepackage{latexsym}
\usepackage{amsfonts}
\usepackage{amssymb}
\usepackage{amsmath}
\usepackage[ruled,vlined]{algorithm2e}
\DeclareUnicodeCharacter{2212}{-}

\usepackage{wrapfig}

\usepackage{mathtools}
\usepackage{booktabs,multicol,multirow}
\usepackage{wasysym}

\newcommand{\newpar}{\vspace{.2in} \noindent}
\usepackage[margin=1.2in]{geometry}

\title{COMP3821 Homework 4}
\author{Jason Qin, z5258237}
\date{April 2020}

\usepackage{natbib}
\usepackage{graphicx}

\begin{document}

\maketitle

\section*{Question 4.}
\paragraph*{4.1} \textbf{Proof of TSP in NP.} \\

\noindent
For all true results of TSP, check whether $m[C_n, C_i] + \sum_{1<i<n} M[C_i, C_{i+1}] \leq k$.
Also check whether $M[C_i, C_{i+1}] \in M$ such that all vertices have been visited once.
These checks can be done in polynomial time (linear). Thus there exists a verification algorithm
to verify the results of TSP in polynomial time.

\paragraph*{4.2} \textbf{Construction of reduction.} \\

\noindent
Consider $E \subseteq V \times V$ for some specific set of edges in a graph $G$. Let $E'$ be the set of edges
of the complete graph $G'$, where all nodes are connected exactly once to every other node. Then, for all
pairs $e \in E$, let those edges weight $0$ in $E'$, and let all other edges be $1$. Hence $M$ is analoguos
to $E'$, i.e. a matrix containing $1$ and $0$'s for the above representation. \\

\noindent
Now consider $k = 0$ as the
length of a possible cycle through all the vertices in $G'$, and so our target construction is $k \leq 0$ for which
whether the TSP is a YES instance.

\paragraph*{4.3} \textbf{Required properties.}  \\

\noindent
The following are selected,

\begin{itemize}
    \item The reduction takes polynomial time.
    \item The reduction is NP-Hard.
    \item The reduction maps YES instances to YES instances.
    \item The reduction maps NO instances to NO instances.
    \end{itemize}


\paragraph*{4.4} \textbf{Proof that requirements are met.} \\

\noindent
It is known that the Hamiltonian Cycle is NP-Complete, so it is iin both NP and NP-Hard. Hence any valid
reduction from the Hamiltonian Cycle to a problem Y in NP will mean Y is also NP-Hard. If Y is the TSP, then
reducing Hamiltonian Cycle to it means TSP is NP-Hard. \\

\noindent
If a hamiltonian cycle exists in $G$, then the reduction in $4.2$ will return a YES instance. This is where
$G'$ has a cycle of length $l = 0$ through all its vertices. Hence we have a directly mapping to a YES instance
in the TSP, since a hamiltonian cycle with length $0$ means there is a solution in TSP for $k \leq 0$. \\

\noindent
Similarly, if no hamiltonian cycle exists in $G$, then it will return a NO instance. This means there is
no cycle of length $k = 0$ passing through all the vertices in $G'$. Hence there is no solution for the
TSP for which $k \leq 0$, and we also get a NO instance. \\

\noindent
It can be seen that the reduction takes no more than $O(n^2)$, i.e. polynomial time, and thus all the requirements
are met. \\

\noindent
Therefore, TSP is NP-Complete. \\

\end{document}
