\documentclass{article}
\usepackage[utf8]{inputenc}
\usepackage{latexsym}
\usepackage{amsfonts}
\usepackage{amssymb}
\usepackage{amsmath}
\usepackage[ruled,vlined]{algorithm2e}
\DeclareUnicodeCharacter{2212}{-}

\usepackage{wrapfig}

\usepackage{mathtools}
\usepackage{booktabs,multicol,multirow}
\usepackage{wasysym}

\newcommand{\newpar}{\vspace{.2in} \noindent}
\usepackage[margin=1.2in]{geometry}

\title{COMP3821 Homework 4}
\author{Jason Qin, z5258237}
\date{April 2020}

\usepackage{natbib}
\usepackage{graphicx}

\begin{document}

\maketitle

\section*{Question 4.}
\paragraph*{4.1} \textbf{Proof of TSP in NP.} \\

\noindent
For all true results of TSP, check whether $m[C_n, C_i] + \sum_{1<i<n} M[C_i, C_{i+1}] \leq k$.
Also check whether $M[C_i, C_{i+1}] \in M$ such that all vertices have been visited once.
These checks can be done in polynomial time (linear). Thus there exists a verification algorithm
to verify the results of TSP in polynomial time.

\paragraph*{4.2} \textbf{Construction of distances.} \\

\noindent


\end{document}
